\documentclass[12pt,a4paper]{article}
\usepackage[utf8]{inputenc}
\usepackage{dsfont} 
\usepackage[polish]{babel}
\usepackage{amsmath}
\usepackage{graphicx}
\usepackage[top=1in, bottom=1.5in, left=1.25in, right=1.25in]{geometry}

\usepackage{subfig}
\usepackage{multirow}
\usepackage{multicol}
\graphicspath{{Imagens/}}
\usepackage{xcolor,colortbl}
\usepackage{float}

\newcommand \comment[1]{\textbf{\textcolor{red}{#1}}}

%\usepackage{float}
\usepackage{fancyhdr} % Required for custom headers
\usepackage{lastpage} % Required to determine the last page for the footer
\usepackage{extramarks} % Required for headers and footers
\usepackage{indentfirst}
\usepackage{placeins}
\usepackage{scalefnt}
\usepackage{xcolor,listings}
\usepackage{textcomp}
\usepackage{color}
\usepackage{verbatim}
\usepackage{framed}

\definecolor{codegreen}{rgb}{0,0.6,0}
\definecolor{codegray}{rgb}{0.5,0.5,0.5}
\definecolor{codepurple}{HTML}{C42043}
\definecolor{backcolour}{HTML}{F2F2F2}
\definecolor{bookColor}{cmyk}{0,0,0,0.90}  
\color{bookColor}

\lstset{upquote=true}

\lstdefinestyle{mystyle}{
	backgroundcolor=\color{backcolour},   
	commentstyle=\color{codegreen},
	keywordstyle=\color{codepurple},
	numberstyle=\numberstyle,
	stringstyle=\color{codepurple},
	basicstyle=\footnotesize\ttfamily,
	breakatwhitespace=false,
	breaklines=true,
	captionpos=b,
	keepspaces=true,
	numbers=left,
	numbersep=10pt,
	showspaces=false,
	showstringspaces=false,
	showtabs=false,
}
\lstset{style=mystyle}

\newcommand\numberstyle[1]{%
	\footnotesize
	\color{codegray}%
	\ttfamily
	\ifnum#1<10 0\fi#1 |%
}

\definecolor{shadecolor}{HTML}{F2F2F2}

\newenvironment{sqltable}%
{\snugshade\verbatim}%
{\endverbatim\endsnugshade}

% Margins
\addtolength{\footskip}{0cm}
\addtolength{\textwidth}{1.4cm}
\addtolength{\oddsidemargin}{-.7cm}

\addtolength{\textheight}{1.6cm}
%\addtolength{\topmargin}{-2cm}

% paragrafo
\addtolength{\parskip}{.2cm}

% Set up the header and footer
\pagestyle{fancy}
\rhead{\hmwkAuthorName} % Top left header
\lhead{\hmwkClass: \hmwkTitle} % Top center header
\rhead{\firstxmark} % Top right header
\lfoot{Cezary Prajwowski} % Bottom left footer
\cfoot{} % Bottom center footer
\rfoot{} % Bottom right footer
\renewcommand{\headrulewidth}{1pt}
\renewcommand{\footrulewidth}{1pt}

    
\newcommand{\hmwkTitle}{Panda Ramen Sushi} % Tytuł projektu
\newcommand{\hmwkDueDate}{\today} % Data 
\newcommand{\hmwkClass}{Technologie chmurowe} % Nazwa przedmiotu
\newcommand{\hmwkAuthorName}{Cezary Prajwowski} % Imię i nazwisko

% trabalho 
\begin{document}
% capa
\begin{titlepage}
    \vfill
	\begin{center}
	\hspace*{-1cm}
	\vspace*{0.5cm}
    \includegraphics[scale=0.55]{imagens/loga.png}\\
	\textbf{Uniwersytet Gdański \\ [0.05cm]Wydział Matematyki, Fizyki i Informatyki \\ [0.05cm] Instytut Informatyki}

	\vspace{0.6cm}
	\vspace{4cm}
	{\huge \textbf{\hmwkTitle}}\vspace{8mm}
	
	{\large \textbf{\hmwkAuthorName}}\\[3cm]
	
		\hspace{.45\textwidth} %posiciona a minipage
	   \begin{minipage}{.5\textwidth}
	   Projekt z przedmiotu technologie chmurowe na kierunku informatyka profil praktyczny na Uniwersytecie Gdańskim.\\[0.1cm]
	  \end{minipage}
	  \vfill
	%\vspace{2cm}
	
	\textbf{Gdańsk}
	
	\textbf{\hmwkDueDate}
	\end{center}
	
\end{titlepage}

\newpage
\setcounter{secnumdepth}{5}
\tableofcontents
\newpage

\section{Opis projektu}
\label{sec:Project}

Nowopowstała, rozwijająca się restauracja z jedzeniem japońskim potrzebuje strony do składania zamówień, wyświetlania menu restauracji oraz do ogłaszania aktualności. Ma być to miejsce pomocne dla klienta jak i dla właściciela restauracji.

\subsection{Opis architektury - 8 pkt}
\label{sec:introduction}

Aplikacja jest wdrożona w klastrze Kubernetes, który odbierając głównym Node'm zapytanie dotyczące stworzenia workloadu, wysyła je do Workera, który następnie za pomocą Dockera uruchamia kontener w sobie. W naszym przypadku mamy 4 takie moduły, SPA (react) dostarczający interfejs użytkowinka, API (Express) interakcja między warstwami, DB (Mongo) do zarządzania bazą danych oraz authorization server (Keycloak) do autoryzacji i autentykacji użytkownika. Każda z nich jako Deployment (zarządzajacy podami) + Service (dający dostęp do komunikacji do/z poda).

\includegraphics[scale=0.45]{imagens/screen.png}\\
\subsection{Opis infrastruktury - 6 pkt}
\label{sec:Users}


Aplikacja działają w kontenerach w klastrze Kubernetes. Z narzędzi używamy Kubernetes do orkiestracji kontenerów, który daje środowisko do odpalenia ich w większej ilości i Docker do tworzenia ich. Z 1Gi pamięci masowej korzysta pod bazy danych (mymongo). 


\subsection{Opis komponentów aplikacji - 8 pkt}
\label{sec:FunctionalConditions}

\begin{itemize}
    \item \textbf{Frontend} 
    \begin{itemize}
        \item Obraz zawierający aplikację React.
        \item Wdrożony jako jedno-replikowy deployment.
        \item Service typu LoadBalancer, dostępny na zewnątrz pod \texttt{localhost:3000}.
        \item Nie zawiera żadnych zmiennych środowiskowych.
    \end{itemize}

    \item \textbf{Backend}
    \begin{itemize}
        \item Aplikacja Express na obrazie \texttt{node:alpine}.
        \item Wdrożona jako auto-skalujący deployment.
        \item Service typu NodePort, dostępny dla innych kontenerów pod URL \texttt{mybackend:8000}.
        \item Zmienne środowiskowe w konfiguracji:
        \begin{itemize}
            \item \texttt{DB\_HOST\_URL} - nazwa serwisu bazy danych.
            \item \texttt{KEYCLOAK\_HOST\_URL} - nazwa serwisu Keycloak.
        \end{itemize}
    \end{itemize}

    \item \textbf{Database (MongoDB)}
    \begin{itemize}
        \item Baza danych oparta na obrazie \texttt{mongo}, z dodanym bazowym \texttt{mongoimportem}.
        \item Wdrożona jako StatefulSet (2 repliki).
        \item Service z nazwą domeny wewnątrz sieci Kubernetes \texttt{mymongo-set}.
        \item Zmienne środowiskowe:
        \begin{itemize}
            \item \texttt{MONGO\_INITDB\_ROOT\_USERNAME} - implementowane dzięki \texttt{mongo-secret}.
            \item \texttt{MONGO\_INITDB\_ROOT\_PASSWORD} - implementowane dzięki \texttt{mongo-secret}.
        \end{itemize}
    \end{itemize}

    \item \textbf{Keycloak}
    \begin{itemize}
        \item Serwer autoryzujący i uwierzytelniający.
        \item Wdrożony jako jedno-replikowy deployment.
        \item Przekazane dane w formie secret dotyczące danych wrażliwych:
        \begin{itemize}
            \item Hasło Keycloak.
            \item Hasło PostgreSQL.
        \end{itemize}
        \item Przekazane dane z ConfigMap:
        \begin{itemize}
            \item Porty.
            \item Hosty.
            \item Nazwy użytkownika.
        \end{itemize}
        \item Service typu LoadBalancer:
        \begin{itemize}
            \item Dostępny lokalnie na \texttt{localhost:8080}.
            \item Dostępny dla innych kontenerów w wewnętrznej sieci na \texttt{keycloak-service:8080}.
        \end{itemize}
    \end{itemize}
    
    \item \textbf{Postgres}
    \begin{itemize}
        \item Baza danych PostgreSQL zamiast defaultowej H2 keycloaka, wdrożona jako Deployment i Service.
    \end{itemize}
\end{itemize}


\subsection{Konfiguracja i zarządzanie - 4 pkt}
\label{sec:NonFunctionalConditions}

Klastry wdrażane są za pomocą plików .yaml, które inicjalizują Deploymenty, Serwisy, configmapy, secrety oraz metryki. Kubelet zajmuje sie zarządzaniem i utrzymywaniem poda. Konfiguracja jest wdrażana za pomocą config-mapów oraz secret`ów. Zarządzanie aplikacją głównie komendami kubectl obsługiwanymi przez kubeleta.

\subsection{Zarządzanie błędami - 2 pkt}
\label{sec:ERD} 
Kubelet zajmuje się sprawdzaniem czy kontener jest zdrowy. 
Pody automatycznie się restartują w razie wystąpienia awarii. Aby dodatkowo to zabezpieczyć do deploymentu backendu dodałem sondy livenessProbe oraz readinessProbe. Pierwsza sprawdza stan aplikacji i jeśli stwierdzi, że aplikacja nie działa poprawnie to ją zrestartuje. Dopóki druga sonda readinessProbe nie potwierdzi, że aplikacja jest gotowa, Kubernetes nie przesyła do niej ruchu, co zapobiega niedziałającym żądaniom.

\subsection{Skalowalność - 4 pkt}
\label{sec:ExamplesSection}

Skalowanie jest wdrożone przy użyciu narzędzia HPA (HorizontalPodAutoscaler) na deploymencie backendu, który jest najbardziej narażony na problemy związane z nagłym przypływem żądań. Polega on na badaniu średniego obciążenia CPU aplikacji, gdy przekroczy 50\% startuje kolejny pod, aby rozłożyć ruch na kilka replik. Maksymalnie ustawione na 10 replik.

\subsection{Wymagania dotyczące zasobów - 2 pkt}
\label{sec:ExampleTables}

Powinna zawierać informacje na temat wymagań dotyczących zasobów dla każdego komponentu aplikacji, takie jak ilość pamięci RAM, CPU, miejsce na dysku, itp. Należy również opisać, jakie są oczekiwania dotyczące wydajności i czasu odpowiedzi dla aplikacji.

Dane zużycia RAM oraz CPU w tabeli pochodzą z komendy kubectl top pods. Mierzone były podczas normalnego użytkowania przez jednego użytkownika strony internetowej. 

\begin{table}[H]
    \centering
    \begin{tabular}{|l|c|c|c|}
        \hline
        \textbf{Serwis} & \textbf{Time (seconds)} & \textbf{CPU}& \textbf{RAM} \\
        \hline
        Frontend & 0.002s  & 3m & 431Mi\\ 
        \hline
        Backend & 0.008s & 3m & 35Mi\\
        \hline
        MongoDB (1x set) & 0.001s & 5m & 217Mi\\
        \hline
        Keycloak & 0.024s & 3m & 635Mi \\
        \hline
    \end{tabular}
    \caption{Czasy odpowiedzi (testowane przez curl adres)}
    \label{tab:component-times}
\end{table}

Mongo rezerwuje 512Mi RAM oraz 500m i ma taki sam limit ze względu na zasadę guaranteed Quality of Service and guaranteed performance.

\begin{table}[H]
    \centering
    \begin{tabular}{|l|c|c|}
        \hline
        \textbf{Komponent} & \textbf{Rezerwacja} & \textbf{Limit} \\
        \hline
        Backend (CPU / RAM) & 100m / 124Mi & 1000m / 1024Mi \\
        \hline
        MongoDB (CPU / RAM) & 500m / 512Mi & 500m / 512Mi \\
        \hline
    \end{tabular}
    \caption{Rezerwacja i limity zasobów dla backendu i MongoDB w Kubernetes}
    \label{tab:resource-limits}
\end{table}
Dodatkowo 1Gi miejsca na dysku dla MongoDB.

\subsection{Architektura sieciowa - 4 pkt}
\label{sec:ExampleResults}



Mikroserwisy komunikują się za pomoca wewnętrznej sieci. Na zewnątrz na adres localhost:3000 za pomocą LoadBalancera jest wystawiona aplikacja React, która czerpie z innych serwisów, wysyłając zapytania routowane przez sieć wewnętrzną klastra na przypisane im domeny. Wykorzystane protokoły to TCP/IP oraz HTTP.
\noindent


\bibliographystyle{amsplain}
\bibliography{references.bib}

https://kubernetes.io/docs/tasks/run-application/horizontal-pod-autoscale-walkthrough/

\nocite{*}

\end{document}